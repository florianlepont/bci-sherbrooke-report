\chapter* {Introduction}
\addcontentsline{toc}{chapter}{Introduction} % Ajout dans la table des matieres
\thispagestyle{fancy}

Une interface Cerveau-Ordinateur ou BCI (Brain Computer Interface) permet de réaliser une communication allant du cerveau vers un système numérique (ordinateur par exemple). Pour cela, on s'appuie sur une étude de l'activité neuronale du cerveau grâce à un système électroencéphalographique (EEG) équipé d'électrodes. Celles-ci sont placées à la surface du crâne afin de capter l'activité électrique du cerveau. Un traitement et une analyse des signaux électriques saisis sont réalisés afin de les traduire en informations exploitables par les systèmes numériques. 

Il existe de nombreuses applications possibles pour ce type de technologie : 

\smallbreak

\begin{itemize}
	\item \textbf{Recherche.} L'étude du cerveau et son activité neuronale, afin d'en comprendre les différents mécanismes de fonctionnement.
	\smallbreak
	\item \textbf{Médicale.} Ce dispositif permet d'aider à comprendre et à guérir certaines maladies (Parkinson, etc.). Il peut également assister les médecins dans le traitement de l'hyperactivité chez certaines personnes, en améliorant leur concentration. Enfin, ce système permet le contrôle de prothèses robotisées pour les personnes amputées.
	\smallbreak
	\item \textbf{Vidéo-ludique.} Les liaisons cerveau-ordinateur sont de plus en plus utilisées dans l'univers des jeux-vidéos. 
\end{itemize}

\smallbreak

Le but de cette étude est de comprendre le fonctionnement d'une liaison Cerveau-Ordinateur et d'appliquer les concepts retenus grâce à la réalisation d'une expérience vidéo-ludique simple (interagir avec un ordinateur). Ce travail est réalisé dans le cadre de notre projet de recherche et de développement (GIN956), lors de la session d'hiver 2015.

La première étape de notre démarche est donc d'étudier la physiologie du cerveau et les principes biologiques qui y sont rattachés. On examinera par la suite les différents systèmes EEG présents sur le marché et on choisira le plus approprié. 
Dans un troisième temps, on propose d'explorer les différentes méthodes existantes de traitement et d'analyse des signaux EEG.
Afin de mettre en application les informations acquises, on réalisera une expérience simple d'interaction. 
Enfin, on propose d'intégrer nos travaux dans le cadre de cours dispensés à l'université de Sherbrooke. Il s'agit de soumettre des travaux pratiques en lien avec les outils que nous utilisons dans le cadre de notre projet, afin de donner un aspect plus concret à ces cours. Voici les cours concernés.  
\smallbreak
\begin{itemize}
	\item \textbf{Cours gradué de techniques avancées en traitement des signaux.} Ce cours enseigne les différentes méthodes de représentation d'un signal dans le domaine spectral (transformée de Fourier, transformation en ondelettes, etc.), ainsi que les outils de filtrage spectral (analyse en composantes principales (PCA), analyse en composantes indépendantes (ICA)). Cours enseigné à la session d'automne. 
	\smallbreak
	\item\textbf{APP d'intelligence probabiliste}. Ce cours s'intéresse aux différentes méthodes de classification, notamment les méthodes paramétriques (Bayesiennes) et non paramétriques (k-moyen, SVM, etc.). Cours enseigné à la session d'automne.
	\smallbreak
	\item \textbf{Cours de neurosciences computationnelles}. Ce cours enseigne les principes de base de la physiologie des réseaux neuronaux et leur modélisation. Il offre également une vision d'ensemble des différents domaines d'applications. Cours enseigné à la session d'hiver.
\end{itemize}

