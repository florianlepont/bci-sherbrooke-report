\chapter* {Conclusion}
\addcontentsline{toc}{chapter}{Conclusion} % Ajout dans la table des matieres
\thispagestyle{fancy}



Nous avons eu la chance de réaliser notre projet de recherche et développement (GIN956) au sein du laboratoire de recherche en neurosciences NECOTIS, sous la tutelle de Jean Rouat. Cette opportunité nous a permis de bénéficier d'une première expérience dans le monde des neurosciences, et plus particulièrement dans la conception d'une interface cerveau-ordinateur. À cette occasion, nous avons pu acquérir de nouvelles connaissances théoriques (traitement de signaux EEG, physiologie du cerveau, etc.), pratiques (utilisation d'un casque EEG, utilisation d'un logiciel de création de liaisons cerveau-ordinateur) et méthodologiques (gestion de projet, autonomie, apprentissage à l'utilisation de \LaTeX, etc.)

Dans un premier temps, nous avons étudié les différents aspects physiologiques du cerveau qui nous seraient utiles dans la réalisation de l'interface cerveau-ordinateur (organisation corticale, différents rythmes du cerveau, etc.). 

Nous avons ensuite réalisé une étude comparative des différents systèmes EEG proposés sur le marché (casques EEG, électrodes, etc). Celle-ci nous a permis d'arrêter notre choix sur le casque EPOC de la société Emotiv. Nous avons également étudié les différents logiciels libres de droit disponibles, permettant de réaliser une liaison cerveau-ordinateur. OpenViBE se trouve être l'outil informatique le plus adapté pour la réalisation de la BCI dans le cadre de notre projet. 

Dans un troisième temps, nous avons étudié différentes techniques de traitement de signaux EEG, comme les méthodes de filtrage temporel, les filtrages spatiaux ou encore les différentes techniques de classification de caractéristiques. 

Nous présentons par la suite le fonctionnement du logiciel OpenViBE et du casque EPOC, ainsi que l'ensemble des outils s'y rattachant. 

A partir de l'ensemble des études réalisées précédemment, on réalise une expérience interactive. Celle-ci s'appuie sur une chaîne de traitement et d'analyse des signaux EEG composée d'un filtrage temporel passe-bande de type Butterworth, un filtrage spatial de type CSP et une classification des caractéristiques des signaux de types LDA.

Enfin, on intégre les différents éléments de notre projet dans le cours de neurosciences computationnelles et de traitement du signal avancé, en s'appuyant sur des travaux pratiques (réalisation d'un filtrage spatial de type PCA/ICA) réalisés sous l'environnement Matlab.
