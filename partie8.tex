\chapter{Bilan du projet}
\thispagestyle{fancy}

\section {Connaissances acquises}

Ce projet captivant, de part par son originalité et son aspect pédagogique, nous a permis de mettre en application nos acquis en traitement du signal, en informatique et en neurosciences. D'un point de vue théorique, la pluridisciplinarité du sujet nous a offert l'opportunité d'acquérir de nouvelles connaissances théoriques (traitement de signaux EEG, physiologie du cerveau, etc.). Celui-ci nous a également permis d'apprendre à utiliser de nouveaux outils, comme un casque électroencéphalographique (EPOC d'Emotiv) ou encore le logiciel OpenViBE, permettant de réaliser des interfaces cerveau-ordinateur. Enfin, c'est véritablement au niveau de la méthodologie de recherche, de la gestion de projet et de l'autonomie que nous pensons avoir le plus progressé. Notons que celui-ci nous a notamment permis d'apprendre à utiliser l'outil de traitement de texte \LaTeX, afin de rédiger un rapport répondant aux normes IEEE.

\section {Problèmes rencontrés}

Notre travail devait initialement s'appuyer sur les travaux réalisés par un étudiant de la session d'automne 2014. Cependant, ses résultats ne nous semblaient pas pertinents dans le cadre de notre projet. De ce fait, nous avons décidé de reprendre le projet depuis le départ (notamment la partie portant sur le choix du matériel et du logiciel, ainsi que les traitements des signaux).

D'un point de vue théorique, les connaissances que nous avions en traitement de signaux EEG et en intelligence artificielle probabiliste étaient quasiment nulles (notamment sur l'aspect filtrage spatial et méthodes de classification). Cependant, nous pensons avoir suffisamment renforcé nos connaissances dans ces domaines.

Nous avons également rencontré certaines difficultés d'ordre pratique, notamment lors de l'intégration de notre projet dans le cours de traitement de signal avancé. Nous souhaitions en effet utiliser des outils mathématiques tel que Matlab ou Python. Cependant, leur intégration avec OpenViBE s'est avérée difficile à mettre en place (intégration de Matlab possible uniquement avec la version 32 bits, utilisation de Python trop complexe dans le cadre d'un cours). Nous sommes parvenus à trouver une solution qui consiste en l'exportation des données d'OpenViBE vers le format CSV, puis en les traitant dans Matlab. Cette manière de gérer les données EEG nous oblige cependant à ne plus réaliser la liaison cerveau-ordinateur en temps réel. Enfin, Nous avons également été confronté à quelques contraintes matérielles. La commande (et la livraison) du casque EPOC en Mars nous a obligé à implémenter l'interface cerveau-ordinateur tardivement (le travail sur l'aspect théorique du sujet a donc été privilégié dans un premier temps).

\section {Pistes d'amélioration et poursuite de nos travaux}
Par manque de temps, nous n'avons malheureusement pas pu intégrer notre expérience dans le cadre du cours d'intelligence artificielle, donné à la session d'automne. Il aurait également été intéressant d'utiliser certains outils de neurosciences computationnelles dans le cadre de notre projet, notamment pour les éléments de traitement du signal comme les filtres spatiaux ou les méthodes de classification (utilisation de réseaux neuronaux avec apprentissage).

